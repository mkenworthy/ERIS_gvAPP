%\documentclass[referee]{aa} % for a referee version
%\documentclass[onecolumn]{aa} % for a paper on 1 column  
%\documentclass[longauth]{aa} % for the long lists of affiliations 
%\documentclass[letter]{aa} % for the letters 
\documentclass{aa}

\usepackage{txfonts}
\usepackage{natbib}

\usepackage{graphicx}

\usepackage{color}
\usepackage{hyperref}
\hypersetup{colorlinks=true,allcolors=[rgb]{0,0,0.8}}


\usepackage{showyourwork}

% the three lines suppress the hyperref 'link empty' warnings
% explanation at: https://tex.stackexchange.com/questions/345764/journal-class-shows-package-hyperref-warning-suppressing-link-with-empty-targe
\makeatletter
\renewcommand*\aa@pageof{, page \thepage{} of \pageref*{LastPage}}
\makeatother

% text highlighting
\usepackage{soul}
\sethlcolor{yellow}



\begin{document} 
\authorrunning{Kenworthy et al.}
\titlerunning{ERIS gvAPP}
  \title{The ERIS gvAPP Coronagraph: theoretical and on-sky performance}

   \author{Matthew A. Kenworthy
          \inst{1}
          \and
          TBD
          \inst{2}
          }

   \institute{Leiden Observatory, University of Leiden,
   PO Box 9513, 2300 RA Leiden, The Netherlands\\
   \email{kenworthy@strw.leidenuniv.nl} \\
   \and 
      Astron Institute, Science Road, NL
}
   \date{Received XXXX; accepted XXXX}

% \abstract{}{}{}{}{} 
% 5 {} token are mandatory
 
  \abstract
  % context heading (optional)
  % {} leave it empty if necessary  
   {}
  % aims heading (mandatory)
   {We describe the gvAPP coroangraph for ERIS.}
  % methods heading (mandatory)
   {On sky observations from the commissioning run enabled characterisation of the gvAPP performance.}
  % results heading (mandatory)
   {Excellent matching with the prescribed pattern.
   %
   Different wavelength performance.}
  % conclusions heading (optional), leave it empty if necessary 
   {}

   \keywords{instrumentation -- coronagraphs}

   \maketitle
%
%-------------------------------------------------------------------

   \section{Introduction}

%\begin{figure}
%%    \begin{centering}
 %   \includegraphics[width=0.5\textwidth]{figures/gvapp_psf.pdf}
 %   \caption{gvAPP psf as seen in ERIS.
%              }
%    \label{fig:erisapp}
%    \script{plot_eris_gvapp_psf.py}
%    \end{centering}
%\end{figure}

\section{Phase pattern design}

Phase pattern

goal IWA, OWA, contrast level

determining the separation of the PSFs

\section{Optical construction of the gvAPP}

Optical layering

Flatness

Wedge

\section{On-sky PSF comparison with model PSF}

\subsection{Compared with cryogenic tests}

Anne Boehle discussion

\subsection{narrow band PSF on-sky}


\subsection{IB on sky}

\subsection{Wavelength performance}


\section{Contrast curves}



%------------------------
\section{Conclusions}\label{sec:conclusion}

\subsection{Impact on future designs for ELT class telescopes}

\begin{acknowledgements}

This research has used the SIMBAD database, operated at CDS, Strasbourg, France \citep{wenger2000}.
%
This work has used data from the European Space Agency (ESA) mission {\it Gaia} (\url{https://www.cosmos.esa.int/gaia}), processed by the {\it Gaia} Data Processing and Analysis Consortium (DPAC, \url{https://www.cosmos.esa.int/web/gaia/dpac/consortium}).
%
Funding for the DPAC has been provided by national institutions, in particular the institutions participating in the {\it Gaia} Multilateral Agreement.
%
To achieve the scientific results presented in this article we made use of the \emph{Python} programming language\footnote{Python Software Foundation, \url{https://www.python.org/}}, especially the \emph{SciPy} \citep{virtanen2020}, \emph{NumPy} \citep{numpy}, \emph{Matplotlib} \citep{Matplotlib}, \emph{emcee} \citep{foreman-mackey2013}, and \emph{astropy} \citep{astropy_1,astropy_2} packages.
%
\end{acknowledgements}

\bibliographystyle{aa}
\bibliography{bib}

\end{document}
